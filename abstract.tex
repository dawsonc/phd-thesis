% From mitthesis package
% Version: 1.01, 2023/06/19
% Documentation: https://ctan.org/pkg/mitthesis
%
% The abstract environment creates all the required headers and footnote. 
% You only need to add the text of the abstract itself.
%
% Approximately 500 words or less; try not to use formulas or special characters
% If you don't want an initial indentation, do \noindent at the start of the abstract

Before robots can be deployed in safety-critical environments, we must be able to verify they they will perform safely, ideally without the risk and expense of real-world testing. A wide variety of formal methods and simulation-driven techniques have been developed to solve this verification problem, but they typically rely on difficult-to-construct mathematical models or else use sample-inefficient black-box optimization methods. Moreover, existing verification methods provide little guidance on how to optimize the system's design to be more robust to the failures they uncover.
%
In this thesis, we develop a suite of methods that accelerate verification and design automation of robots and other autonomous systems by using program analysis tools such as automatic differentiation and probabilistic programming to automatically construct mathematical models of the system under test. In particular, we make the following contributions. First, we use automatic differentiation to develop a flexible, general-purpose framework for end-to-end design automation and statistical safety verification for autonomous systems. Second, we improve the sample efficiency of end-to-end optimization using adversarial optimization to falsify differentiable formal specifications of desired robot behavior. Third, we provide a novel reformulation of the design and verification problem using Bayesian inference to predict a more diverse set of challenging adversarial failure modes. Finally, we present a data-driven method for root-cause failure diagnosis, allowing system designers to infer what factors may have contributed to failure based on noisy data from real-world deployments.
%
We apply the methods developed in this thesis to a range of challenging problems in robotics and cyberphysical system. We demonstrate the use of our design and verification framework to optimize spacecraft trajectory and control systems, multi-agent formation and communication strategies, vision-in-the-loop controllers for autonomous vehicles, and robust generation dispatch for electrical power systems, and we apply our failure diagnosis tool on real-world data from scheduling failures in a nationwide air transportation network.